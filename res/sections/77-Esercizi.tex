\section{Esercizi}

\subsection{Domande a risposta multipla}

In the estimated linear regression model $\hat{Y} = \hat{\beta_0} + \hat{\beta_1}X$ we have $\hat{\beta_1} = 0$. Thus:

\begin{itemize}
\item \textbf{$R^2 = 0$} - risposta esatta
\item $R^2 = 1$
\item $R^2 = -1$
\item none of the above
\end{itemize}

In the hypothesis testing, the observed significance level (\textbf{p-value}) is:

\begin{itemize}
\item between 0 and $+\infty$
\item between -1 and 1
\item the type II error
\item \textbf{none of the above} - risposta esatta
\end{itemize}

In a linear regression model, the accuracy of the least squares estimates is measured using:

\begin{itemize}
\item \textbf{standard error} - risposta esatta
\item correlation
\item bias
\item sum of the residuals
\end{itemize}

Errors in the linear regression model are assumed to be:

\begin{itemize}
\item with mean equal to 1
\item with variance equal to 1
\item \textbf{incorrelated with the covariates} - risposta esatta
\item incorrelated with Y
\end{itemize}

Il livello di significatività osservato (detto anche p-value) è:

\begin{itemize}
\item \textbf{una probabilità} - risposta esatta
\item l'errore di secondo tipo
\item una variabile casuale
\item una misura del legame lineare tra X e Y
\end{itemize}

Il residual standard error (RSE) per un modello $Y = \beta_0 + \beta_1 + \epsilon$
con errori che si assumono $N(0,\sigma^2)$ e che viene stimato ai minimi quadrati è:

\begin{itemize}
\item la stima di $\beta_1$
\item la stima della media degli errori
\item \textbf{la stima di $\sigma$} - risposta esatta
\item il p-value associato al test di bontà di adattamento del modello
\end{itemize}

In un modello di regressione lineare, il problema della multicollinearità deriva da:

\begin{itemize}
\item bassa correlazione tra gli errori $\epsilon$ e la risposta
\item bassa correlazione tra tutte le esplicative
\item \textbf{alta correlazione tra almeno due esplicative} - risposta esatta
\item bassa correlazione tra almeno un errore $\epsilon$ e le esplicative
\end{itemize}

\subsection{Note}

Le variabili dummy create da R per variabili qualitative seguono l'ordinamento lessicografico.\\

Il principio di gerarchia stabilisce che se è presente un termine interazione allora anche i termini
che lo compongono devono essere presenti nel modello.















