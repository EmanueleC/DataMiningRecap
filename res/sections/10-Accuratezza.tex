\section{Accuratezza}
\[ E(Y-\hat{Y})^2 = [f(X) - \hat{f}(X)]^2 + Var(\epsilon) \]
dove:
\begin{itemize}
 \item $[f(X) - \hat{f}(X)]^2$ \`e l'errore riducibile dipendente dal 
modello, diminuisce al migliorare della stima;
 \item $Var(\epsilon)$ \`e l'errore irriducibile, dipende da variazioni non 
misurabili che influiscono sulla variabilit\`a di $Y$.
\end{itemize}
L'accuratezza si usa soprattutto durante le previsioni, dove non interessa la forma della funzione $\hat{f}$, ma che valga $\hat{Y} = \hat{f}(x)$. $E(Y-\hat{Y})^2$ non può mai scendere sotto $Var(\epsilon)$ dato che non siamo in grado di ridurre questo termine. Di conseguenza, un valore di accuratezza vicino a $Var(\epsilon)$ è buono perché significa che la stima $\hat{f}$ è proprio uguale a $f$.
