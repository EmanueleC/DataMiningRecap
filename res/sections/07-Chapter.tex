\section{Risorse utili}

\subsection{Materiale didattico}

Libro di testo:
\url{http://www-bcf.usc.edu/~gareth/ISL/ISLR\%20Seventh\%20Printing.pdf}

\subsection{Definizioni}

\df{X}{ha un sacco di nomi\dots input variables/predictors/indipendent
variables/features/covariates/explicative.}\\

\df{Y}{response variable/dependent variable.}\\

\df{MSE}{errore quadratico medio. Può essere calcolato sia sui dati di training
o sui dati di test.}\\

\df{$\sigma$}{deviazione standard.}\\

\df{$\overline{x}$}{quando una variabile è sopralineata indica una media.}\\

\df{Distribuzione normale}{distribuzione di probabilità continua, spesso usata
come prima approssimazione per descrivere variabili casuali a valori reali
che tendono a concentrarsi attorno a un singolo valor medio.
Il grafico della funzione di densità di probabilità associata è simmetrico e
ha una forma a campana, nota come campana di Gauss.
La distribuzione normale dipende da due parametri, la media $\mu$ e la varianza
$\sigma^2$, ed è indicata tradizionalmente con: $N(\mu,\sigma^2)$.}\\

\df{Gradi di libertà}{i gradi di libertà di una variabile aleatoria o di una statistica
in genere esprimono il numero minimo di dati sufficienti a valutare la
quantità d'informazione contenuta nella statistica.}\\

\df{Quantile}{in statistica il quantile di ordine $\alpha$ o $\alpha$-quantili
(con $\alpha$ un numero reale nell'intervallo [0,1]) è un valore $q_\alpha$ che
divide la popolazione in due parti, proporzionali ad $\alpha$ e $1-\alpha$ e
caratterizzate da valori rispettivamente minori e maggiori di $q_\alpha$.
In particolare il quantile di ordine 0 è un qualunque valore inferiore al
minimo della popolazione; similmente il quantile di ordine 1 è un qualunque
valore superiore al massimo della popolazione.}
