\section{Metodo di regressione lineare semplice}
Si è interessati a comprendere come la variabile Y sia influenzata dalla X.

Y è funzione di X se ad ogni valore di X corrisponde un solo valore di Y.

La relazione funzionale è lineare se possiamo scrivere:

\[ Y = \beta_0 + \beta_1 X + \epsilon \]
dove, nel caso del criterio dei minimi quadrati:
\begin{itemize}
 \item $\beta_1$, esplicativa, $\hat{\beta_1} = 
\frac{COV(X,Y)}{VAR(X)} = \frac{\sigma_{XY}}{\sigma^2_{X}}$;
 \item $\beta_0$, intercetta, $\hat{\beta_0} = \overline{y} - \hat{\beta_1} 
\overline{x}$ \\
  con:
  \begin{itemize}
   \item $\overline{x} = \sum_{i=1}^n x_i/n $;
   \item $\overline{y} = \sum_{i=1}^n y_i/n $;
   \item $n$ \`e la grandezza del campione.
  \end{itemize}

 \item $\epsilon$ rappresenta il termine di errore (sempre presente).
\end{itemize}

