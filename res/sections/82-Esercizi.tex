\newpage
\section{Esercizi}

\subsection{Domande a risposta multipla}

In the estimated linear regression model
$\hat{Y} = \hat{\beta_0} + \hat{\beta_1}X$ we have $\hat{\beta_1} = 0$. Thus:

\begin{itemize}
\item \textbf{$R^2 = 0$} - risposta esatta
\item $R^2 = 1$
\item $R^2 = -1$
\item none of the above
\end{itemize}

In the hypothesis testing, the observed significance level (\textbf{p-value}) is:

\begin{itemize}
\item between 0 and $+\infty$
\item between -1 and 1
\item the type II error
\item \textbf{none of the above} - risposta esatta
\end{itemize}

In a linear regression model, the accuracy of the least squares estimates is measured using:

\begin{itemize}
\item \textbf{standard error} - risposta esatta
\item correlation
\item bias
\item sum of the residuals
\end{itemize}

Errors in the linear regression model are assumed to be:

\begin{itemize}
\item with mean equal to 1
\item with variance equal to 1
\item \textbf{incorrelated with the covariates} - risposta esatta
\item incorrelated with Y
\end{itemize}

Il livello di significatività osservato (detto anche p-value) è:

\begin{itemize}
\item \textbf{una probabilità} - risposta esatta
\item l'errore di secondo tipo
\item una variabile casuale
\item una misura del legame lineare tra X e Y
\end{itemize}

Il residual standard error (RSE) per un modello $Y = \beta_0 + \beta_1X + \epsilon$
con errori che si assumono $N(0,\sigma^2)$ e che viene stimato ai minimi quadrati è:

\begin{itemize}
\item la stima di $\beta_1$
\item la stima della media degli errori
\item \textbf{la stima di $\sigma$} - risposta esatta
\item il p-value associato al test di bontà di adattamento del modello
\end{itemize}

In un modello di regressione lineare, il problema della multicollinearità deriva da:

\begin{itemize}
\item bassa correlazione tra gli errori $\epsilon$ e la risposta
\item bassa correlazione tra tutte le esplicative
\item \textbf{alta correlazione tra almeno due esplicative} - risposta esatta
\item bassa correlazione tra almeno un errore $\epsilon$ e le esplicative
\end{itemize}

Il residuo in un modello in cui $y_1$ indica l’osservazione i-esima della variabile
risposta e $\hat{y_i}$ la previsione basata sul modello è pari a: $y_i - \hat{y_i}$\\

Sia dato il modello di regressione lineare semplice $Y = \beta_0 + \beta_1X + \epsilon$
Sia 1.325 la stima ai minimi quadrati di $\beta_1$ e 0.23 il suo standard error. 
La statistica test per la verifica d'ipotesi per $H_0 : \beta_1 \neq 1$ contro
l'alternativa bilaterale assume valore:

Occorre svolgere i calcoli: (1.325 - 1)/0.23 = 0.325/0.23 = 1.413043478\\

Per stimare il modello di regressione lineare $Y = \beta_0 + \beta_1X + \epsilon$ sulla
base di n osservazioni il criterio dei minimi quadrati minimizza
$\sum_{i=1,..,n} y_i - \beta_0 - \beta_1X$

\subsection{Ulteriori note}

Le variabili dummy create da R per variabili qualitative seguono
l'ordinamento lessicografico.\\

Il principio di gerarchia stabilisce che se è presente un termine
interazione allora anche i termini che lo compongono devono essere presenti
nel modello.\\

Scrivere nelle assunzioni che, se il numero di osservazioni è grande (>30),
allora nel calcolo degli intervalli di confidenza e nel test delle ipotesi
si usa la distribuzione normale (scriverlo anche se si dà per scontato).\\

$R^2$ tende a variare sempre un po' se si aggiungono/tolgono covarianti a un
modello. Occorre osservare i p-value associati ai covarianti per stabilire se
sono significativi o no.\\

Il grafico con la distanza di Cooq mostra l'influenza che hanno le osservazioni/
punti. Se ci sono degli outliers (punti che vanno fuori dalla distanza di Cooq)
occorre toglierli dal modello e valutare cosa succede.\\

Nel modello di regressione lineare multipla l'ipotesi di test di riferimento
(f-statistic riportata da R) è quella valutata su tutti i parametri dei
covarianti. Se si vuole confrontare due modelli innestati, l'ipotesi fa riferimento
solo a parametri dei covarianti aggiunti/tolti. $RSS_0$ fa riferimento al modello
con meno parametri... F non può essere negativo. Solitamente quando si aggiungono
parametri a un modello questo diventa più preciso e RSS diminuisce.\\

Rileggere sempre bene le domande così da vedere se si stanno tralasciando 
risposte, calcoli o spiegazioni/assunzioni ecc...
