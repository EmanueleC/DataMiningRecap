\newpage

\section{Regularization}

Come fare quando p è molto grande rispetto a n? p>>n?

Ci sono diversi metodi per far fronte a questo problema

\begin{itemize}
 \item Model selection
 \item Regularization
 \item dimension reduction methods
\end{itemize}

\subsection{Penalized likelihood}

$Q(\beta|X,y) = L(\beta|X,y) + P_{\lambda}(\beta)$

$P_{\lambda}(\beta)$ è il termine di penalizzazione dei parametri
che risultano irrealistici (li fa avvicinare a 0, intercetta
esclusa).

Le variabili devono essere normalizzate prima di applicare questo
metodo. $\lambda$ è il \textbf{parametro di tuning} che controlla
il trade-off tra L e P.

La normalizzazione di una variabile X con valore atteso $\mu$ e
varianza $\sigma$ avviene in questo modo:

\begin{equation}
\frac{X-\mu}{\sigma}
\end{equation}

\subsection{Ridge regression}

\begin{equation}
Q(\beta|X,y) = RSS + \lambda \sum_{j = 1}^p \beta_j ^2
\end{equation}

Anche in questo caso il valore di $\lambda$ influisce
il modello. Se è troppo grande, i coefficienti si avvicinano
molto a 0, altrimenti si ottiene lo stesso effetto della stima
dei minimi quadrati.

La ridge regression si usa in caso di grande variabilità nei
dati.
