\section{MSE}

\subsection{Training MSE}
Il \textit{training MSE}\footnote{Mean Squared Error} \`e l'errore quadratico 
medio calcolato sul training data, e corrisponde a:

\[ MSE = \frac{1}{n} \sum_{i=1}^n \{y_i - \hat{f}(x_i)\}^2 \]

\begin{itemize}
 \item $\hat{f}(x_i)$ indica la predizione che $\hat{f}$ da per l'osservazione 
$i$-esima
\end{itemize}


Questo valore indica quanto il valore per l'osservazione data \`e vicina al suo 
valore vero. Se l'MSE \`e piccolo significa che il valore predetto \`e molto 
vicino alla vera risposta.

\subsection{Test MSE}

Pi\`u importante \`e l'MSE applicato al \textit{test set}, che si calcola con:

\[ Ave\{y_0 - \hat{f}(x_0)\}^2 \]

Dove:
\begin{itemize}
 \item $Ave$ sta per \textit{average}, ovvero la media, che viene calcolata su 
molte osservazioni ($y_0$, $x_0$)
\end{itemize}